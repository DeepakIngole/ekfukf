%%%%%%%%%%%%%%%%%%%%%%%%%%%%%%%%%%%%%%%%%%%%%%%%%%%%%%%%%%%%%%%%%%%%%%%%%%%%%%
%
\chapter{Introduction}
%
%%%%%%%%%%%%%%%%%%%%%%%%%%%%%%%%%%%%%%%%%%%%%%%%%%%%%%%%%%%%%%%%%%%%%%%%%%%%%%

The term optimal filtering refers to methodology used for estimating
the {\it state} of a time varying system, from which we observe
indirect noisy measurements. The state refers to the physical state,
which can be described by dynamic variables, such as position,
velocity and acceleration of a moving object. The noise in the
measurements means that there is a certain degree of uncertainty in
them.  The dynamic system evolves as a function of time, and there is
also noise in the dynamics of system, {\it process noise}, meaning
that the dynamic system cannot be modelled entirely deterministically.
In this context, the term filtering basically means the process of
filtering out the noise in the measurements and providing an optimal
estimate for the state given the observed measurements and the
assumptions made about the dynamic system.

This toolbox provides basic tools for estimating the state of a linear
dynamic system, the Kalman filter, and also two extensions for it, the
extended Kalman filter (EKF) and unscented Kalman filter (UKF), both
of which can be used for estimating the states of nonlinear dynamic
systems. Also the smoother counterparts of the filters are provided.
Smoothing in this context means giving an estimate of the state of the
system on some time step given all the measurements including ones
encountered after that particular time step, in other words, the
smoother gives a smoothed estimate for the history of the system's
evolved state given all the measurements obtained so far.

This documentation is organized as follows: 
\begin{itemize}

\item First we briefly introduce the concept of discrete-time state
  space models. After that we consider linear, discrete-time state
  space models in more detail and review Kalman filter, which is the
  basic method for recursively solving the linear state space
  estimation problems. Also Kalman smoother is introduced. After that
  the function of Kalman filter and smoother and their usage in this
  toolbox in demonstrated with one example (CWPA-model).

\item Next we move from linear to nonlinear state space models and
  review the extended Kalman filter (and smoother), which is the
  classical extension to Kalman filter for nonlinear estimation. The
  usage of EKF in this toolbox is illustrated exclusively with one
  example (Tracking a random sine signal), which also compares the
  performances of EKF, UKF and their smoother counter-parts.

\item After EKF we review unscented Kalman filter (and smoother),
  which is a newer extension to traditional Kalman filter to cover
  nonlinear filtering problems. The usage of UKF is illustrated with
  one example (UNGM-model), which also demonstrates the differences
  between different nonlinear filtering techniques.

\item We extend the concept of sigma-point filtering by studying
  other non-linear variants of Kalman filters. The Gauss-Hermite
  Kalman filter (GHKF) and third-order symmetric Cubature Kalman
  filter (CKF) are presented at this stage.

\item To give a more thorough demonstration to the provided methods
  two more classical nonlinear filtering examples are provided
  (Bearings Only Tracking and Reentry Vehicle Tracking).

\item In chapter~\ref{ch:IMM} we shortly review the concept multiple model
  systems in general, and in sections~\ref{ch:IMM}.1 and \ref{ch:IMM}.2 we take a look at
  linear and non-linear multiple model systems in more detail. We also
  review the standard method, the Interacting Multiple Model (IMM)
  filter, for estimating such systems. It's usage and function is
  demonstrated with three examples.

%\item Lastly we list and describe briefly all the functions included
%  in the toolbox.

\end{itemize}

Details of the toolbox functions can be found on the toolbox web page, or in
Matlab by typing \texttt{help <function name>}. The mathematical notation used in this document follows the notation used in \citep{Sarkka:2006}.


