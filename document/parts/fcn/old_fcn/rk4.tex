% This function reference sheet is automatically generated
% by a Matlab script. The information shown here is based on
% the actual function description in Matlab. Please, if this
% description contains errors, edit the original function
% rather than just fixing the problem here.
% --- Script by Arno Solin (Oct 4, 2010)


\subsubsection*{rk4}
\label{function:rk4}

\noindent
\begin{tabular*}{\textwidth}{@{\extracolsep{\fill}} | l l p{0.7\textwidth} |  }
\hline
\multicolumn{3}{| p{0.9\textwidth} |}{\bf \texttt{rk4}} \\
\multicolumn{3}{| p{0.9\textwidth} |}{
    Perform one fourth order Runge-Kutta iteration step
    for differential equation} \\
\hline
\textbf{Syntax:} & 
  \multicolumn{2}{ p{0.7\textwidth} |}{\texttt{
    [x,Y] = rk4(f,dt,x,[P1,P2,P3,Y])} } \\
\hline
\multirow{7}{*}{\bf Input:}
 & \texttt{f} & Name of function in form f(x,P(:)) or
         inline function taking the same parameters.
         In chained case the function should be f(x,y,P(:)). \\
 & \texttt{dt} & Delta time as scalar. \\
 & \texttt{x} & Value of x from the previous time step. \\
 & \texttt{P1} & Values of parameters of the function at initial time t
         as a cell array (or single plain value). Defaults to empty
         array (no parameters). \\
 & \texttt{P2} & Values of parameters of the function at time t+dt/2 as
         a cell array (or single plain value). Defaults to P1 and
         each empty (or missing) value in the cell array is replaced
         with the corresponding value in P1. \\
 & \texttt{P3} & Values of parameters of the function at time t+dt.
         Defaults to P2 similarly to above. \\
 & \texttt{Y} & Cell array of partial results y1,y2,y3,y4 in the RK algorithm
         of the second parameter in the interated function. This can be
         used for chaining the integrators. Defaults to empty. \\
\hline
\multirow{2}{*}{\bf Output:}
 & \texttt{x} & Next value of X \\
 & \texttt{Y} & Cell array of partial results in Runge-Kutta algorithm. \\
\hline
\end{tabular*}